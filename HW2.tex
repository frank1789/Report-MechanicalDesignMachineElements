\chapter{Homework 2}
\section{Introduction}
Using beam elements, build up a FE model of the hook shown in the Figure. Determine the
maximum displacement, the maximum bending stress, shear and bending moment diagrams. Compare the results obtained considering the cross-sections 1 and 2 shown
below. Discuss the need of mesh refinement.\\
\noindent Data:\\
$F = 20 \, kN \quad E = 205 \, GPa \quad \upsilon=0.3$\\
\begin{figure}[!h]
\centering
\subfloat{\includegraphics[width=.30\textwidth]{/imgHW2/HW2}} \quad
\subfloat{\includegraphics[width=.45\textwidth]{/imgHW2/HW2-1}} 
\end{figure}
\section{Approach the problem}
For this problem we are first costritute the two custom cross sections, in the figures \ref{img:HW2-KeySection}, \ref{img:HW2-AREASection} and \ref{img:HW2-MeshSection}; can be observed the construction phases then a free mesh is applied to both stat and saved in their respective files.
\begin{figure}[!h]
\centering
\subfloat[][Section 1\label{img:HW2_KEYsec1}]{\includegraphics[width=.65\linewidth]{imgHW2/Part1/KEYPOINT-SEC1}}\,
\subfloat[][Section 2\label{img:HW2_KEYsec2}]{\includegraphics[width=.65\linewidth]{imgHW2/Part2/KEYPOINT-SEC2}}
\caption{Keyponit Section}
\label{img:HW2-KeySection}
\end{figure}
\begin{figure}[!h]
\centering
\subfloat[][Section 1\label{img:HW2_GEOMsec1}]{\includegraphics[width=.8\linewidth]{imgHW2/Part1/GEOM-SEC1}}\,
\subfloat[][Section 2\label{img:HW2_GEOMsec2}]{\includegraphics[width=.8\linewidth]{imgHW2/Part2/GEOM-SEC2}}
\caption{Geometry of sections}
\label{img:HW2-GeomSection}
\end{figure}
\begin{figure}[!h]
\centering
\subfloat[][Section 1\label{img:HW2_AREAsec1}]{\includegraphics[width=.8\linewidth]{imgHW2/Part1/AREA-SEC1}}\,
\subfloat[][Section 2\label{img:HW2_AREAsec2}]{\includegraphics[width=.8\linewidth]{imgHW2/Part2/AREA-SEC2}}
\caption{Area of sections}
\label{img:HW2-AREASection}
\end{figure}
\pagebreak\\
\begin{figure}[!h]
\centering
\subfloat[][Section 1\label{img:HW2_sec1}]{\includegraphics[width=.8\linewidth]{imgHW2/Part1/MESH-SEC1}}\,
\subfloat[][Section 2\label{img:HW2_sec2}]{\includegraphics[width=.8\linewidth]{imgHW2/Part2/MESH-SEC2}}
\caption{Meshed Section}
\label{img:HW2-MeshSection}
\end{figure}\noindent
In this case it is defined the geometry of the hook through the use of \emph{keypoints} and subsequently connected by lines, as ahow in fiugure \ref{img:HW2-Geometry}.
Then the mesh using the section $n^{\circ}$1 was created in the previuos step and then the section $n^{\circ}$2.\\
For the section $n^{\circ} 1$ the result in figure \ref{img:HW2_sec1}, while section $n^{\circ} 2$ in picture \ref{img:HW2_sec2}.
\begin{figure}[!h]
\centering 
\subfloat[][Keypoint]{\includegraphics[width=.8\linewidth]{imgHW2/Part1/Keypoint-SEC1-mm-4}}\,
\subfloat[][Full geometry]{\includegraphics[width=.8\linewidth]{imgHW2/Part2/GEOM-SEC2-mm-4}}
\caption{Geometry's hook}
\label{img:HW2-Geometry}
\end{figure}
\begin{figure}[!h]
\centering 
\subfloat[][Section 1]{\includegraphics[width=.8\linewidth]{imgHW2/Part1/FORCE-SEC1-mm-4}}\,
\subfloat[][Section 2]{\includegraphics[width=.8\linewidth]{imgHW2/Part2/FORCE-SEC2-mm-4}}
\caption{Hook meshed and costraint}
\label{img:HW2-Mesh+Costraint}
\end{figure}
\section{Result}
In conclusion is shown in the table of performance comparison of the two cross section.\\
\begin{table}[!h]
\centering
\begin{tabular}{lcccc}
\hline
       Area  	& Displacement & maximum bending stress	 & minimum bending stress\\
    $mm^2$ &	 $mm$			& $MPa$				& $MPa$\\
\hline
       	1000	&	2,59533		&283,82				&-324,366\\
\hline
\end{tabular}
\caption{Recap section 1}
\label{table:HW2:RecapSec1}
\end{table}
\begin{table}[!h]
\centering
\begin{tabular}{cccc}
\hline
       Area  & Displacement 	& maximum 	bending stress	& minimum bending stress\\
    $mm^2$ &	 $mm$			& $MPa$				& $MPa$\\
\hline
    658,84	&	2,8004			&	318,819			&	-318,819\\
\hline
\end{tabular}
\caption{Recap section 2}
\label{table:HW2:RecapSec2}
\end{table}
\begin{enumerate}
\item It is observed in the first section a greater deformation despite the area is larger in size when compared with the second.
\item Whereas the difference in the area of the sections, the displacement of the section 2 is similar to the first.
\item Section 2 does not show differences in the stress since Section 1.
\end{enumerate}
In conclusion it is observed a better performance of the section 2 to equal forces applied.
Subsequently, the analysis is repeated increased the division into examination number generating, therefore, a more dense mesh from which it is observed that the obtained have results are very close, in the table \ref{tab:HW2-ref1}, \ref{tab:HW2-ref2}.
\begin{table}[!h]
\centering

\begin{tabular}{ccccc}
\hline
size 	& section	& Bending Stress min 	&	bending stress max	&	displacement\\
		&			& $MPa$					& 	$MPa$				& $mm$\\
\hline
1		&	1		&-324,327				&	283,786				&	2,59533\\
4		&	1		&-324,366				&	283,82				&	2,59533\\
7		&	1  		& -324,45				&	283,894				&	2,59533\\
10		&	1		&-324,584				&	284,011				&	2,59533\\
16		&	1		&-324,989				&	284,365				&	2,59533\\
22		&	1		&-325,36				&	284,69				&	2,59533\\
\hline
\end{tabular}

\caption{Sensibility result to mesh refiniment}
\label{tab:HW2-ref1}
\end{table}
\begin{table}[!h]
\centering
\begin{tabular}{ccccc}
\hline
size 	& section	& Bending Stress min 	&	bending stress max	&	displacement\\
		&			& $MPa$					& 	$MPa$				& $mm$\\
\hline
1		&	2		&	-318,78				&	318,78				&	2,8004\\
4		&	2		&	-318,819			&	318,819				&	2,8004\\
7		&	2  		&	-318,901			&	318,901				&	2,8004\\
10		&	2		&	-319,033			&	319,033				&	2,8004\\
16		&	2		&	-319,431			&	319,431				&	2,8004\\
22		&	2		&	-319,796			&	319,796				&	2,8004\\
\hline
\end{tabular}
\caption{Sensibility result to mesh refiniment}
\label{tab:HW2-ref2}
\end{table}
\pagebreak\\
The following pictures shows the results obtained from the simulation.
\begin{figure}[!h]
\centering
\subfloat[][Section 1]{\includegraphics[width=.65\linewidth]{imgHW2/Part1/Displacemt-SEC1-mm-4}}\,
\subfloat[][Section 2]{\includegraphics[width=.65\linewidth]{imgHW2/Part2/Displacemt-SEC2-mm-4}}
\label{img:HW2-Displacemt}
\caption{Result Displacement diagram}
\end{figure}
\begin{figure}[!h]
\centering 
\subfloat[][Section 1]{\includegraphics[width=.8\linewidth]{imgHW2/Part1/NODALSOLUTION-SEC1-mm-4}}\,
\subfloat[][Section 2]{\includegraphics[width=.8\linewidth]{imgHW2/Part2/NODALSOLUTION-SEC2-mm-4}}
\label{img:HW2-NodalSolu}
\caption{Result nodal solution diagram}
\end{figure}
\begin{figure}[!h]
\centering 
\subfloat[][Section 1]{\includegraphics[width=.8\linewidth]{imgHW2/Part1/BENDINMAX-SEC1-mm-4}}\,
\subfloat[][Section 2]{\includegraphics[width=.8\linewidth]{imgHW2/Part2/BENDINMAX-SEC2-mm-4}}
\label{img:HW2-bendingmax}
\caption{Result max bending diagram}
\end{figure}
\begin{figure}[!h]
\centering 
\subfloat[][Section 1]{\includegraphics[width=.8\linewidth]{imgHW2/Part1/BENDINGMIN-SEC1-mm-4}}\,
\subfloat[][Section 2]{\includegraphics[width=.8\linewidth]{imgHW2/Part2/BENDINGMIN-SEC2-mm-4}}
\label{img:HW2-bendingmin}
\caption{Result minimum bending stress diagram}
\end{figure}
\begin{figure}[!h]
\centering 
\subfloat[][Section 1]{\includegraphics[width=.8\linewidth]{imgHW2/Part1/MOMENT-SEC1-mm-4}}\,
\subfloat[][Section 1]{\includegraphics[width=.8\linewidth]{imgHW2/Part1/MOMENT-SEC1-mm-4}}
\label{img:HW2-Moment}
\caption{Result Moment diagram}
\end{figure}
\begin{figure}[!h]
\centering
\subfloat[][Section 1]{\includegraphics[width=.8\linewidth]{imgHW2/Part1/SHEAR-SEC1-mm-4}}\,
\subfloat[][Section 2]{\includegraphics[width=.8\linewidth]{imgHW2/Part2/SHEAR-SEC2-mm-4}}
\label{img:HW2-Shear}
\caption{Result shear diagram}
\end{figure}
\section{Command list}
\begin{multicols}{2}
\scriptsize
\lstinputlisting[language=APDL, style=apdl-modified]{CListHW2-Sec1.txt}
\lstinputlisting[language=APDL, style=apdl-modified]{CListHW2-Sec2.txt}
\lstinputlisting[language=APDL, style=apdl-modified]{CommandListHW2-SEC1.txt}
\normalsize
\end{multicols}