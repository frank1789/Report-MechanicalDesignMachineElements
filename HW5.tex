\chapter{Homework 5}
\begin{minipage}{.70\textwidth}
\centering
\includegraphics[width=0.8\linewidth]{imgHW5/HW5}
\end{minipage}
\begin{minipage}{.70\textwidth}
\begin{tabular}{l}
        DATA:\\
        Material: steel;\\
        Pressure: $P_{i}= 10$ \textsc{bar};
\end{tabular}
\end{minipage}\\\\
The T pipe connector analyzed in HW 4 is now filletted at the junction between the two pipes to reduce the stress concentration factor. Using brick elements, build a submodel able to estimate the stress distribution along the periphery of the pipes' junction on the base of the displacement field computed with the shell model developed in HW 4. It is required to:
\begin{enumerate}
\item check for the sensitivity of the results upon the location of the cutting boundaries of the submodel.
\end{enumerate}
\section{Approach the problem}
For this problem, first designing the pattern seen in homework 4 saving the results and after that we realize the submodel.
Using the same work surfaces extrude the two vertical and horizontal cylinders, then they eliminate unneeded volumes. The result as show in figure \ref{img:HW5_GEOMsec1}.
It builds the fillet eliminating the junction that was present in the previous step, figure \ref{img:HW5_GEOMsec3}.
\begin{figure}[!h]
\centering
\subfloat[][First step construnction\label{img:HW5_GEOMsec1}]{\includegraphics[width=.8\linewidth]{imgHW5/HW5-Geometry_1_0}}\,
\subfloat[][Second step construction\label{img:HW5_GEOMsec3}]{\includegraphics[width=.8\linewidth]{imgHW5/HW5-Geometry_2_0}}
\caption{3D Geometry of T pipe}
\label{img:HW5-GeomSection}
\end{figure}\\
The model is now trimmed to the sub pattern using two planes: one through the vertical plane; the second is horizontal. Finally the separate volumes are united in a single body.
It realizes a free mesh with type elements \textsc{solid186}, assigned size of the elements is equal one millimeter; such as to ensure that there are at least two elements in the smallest thickness that makes up the vertical cylinder, the result is observable in fig. \ref{img:HW5_Mesh}.
\begin{figure}[!h]
\centering
\includegraphics[width=.8\linewidth]{imgHW5/HW5-MeshDeatil_1_125}
\caption{Mesh model}
\label{img:HW5_Mesh}
\end{figure}\\
At this point, saving the coordinates of the nodes that are located on the surfaces generated from pattern cutting, for later use to define the conditions of the problem outline. 
The results solution's Homework 4 are recalled and through interpolation is assigned a shift on the nodes that you have saved the coordinates above. Finally applies the same pressure of 10 \textsc{bar} and constraints.
\section{Result}
The study is carried out by varying the distance of the cut boundaries in such a way as not to modify the structure of the problem. defining a sub small model and then increase its size. The analysis is summarized, in the table \ref{tab:HW5_iter}, where the cutting distances are quoted with respect to the junction.\\
\begin{table}[htb]
\centering
\begin{tabular}{cccc}
\hline
       iter&   Vertical cut  boundaries  &     Radius cut boundaries\\ 
       		 &		$mm$                          &       $mm$\\     
\hline
        1	&		30,0000   &    65,0000\\
        2	&    	35,0000   &    70,0000\\
        3 &   	40,0000   &    75,0000\\
        4 &    	45,0000   &    80,0000\\
        5 &    	50,0000   &    85,0000\\
        6 &    	55,0000   &    90,0000\\
\hline
\end{tabular}
\caption{Cut boundaries distance}
\label{tab:HW5_iter}
\end{table}
\newpage
\noindent The results satisfied the equivalent stress, according to Von Mises, obtained under varying cut boundaries are shown in figures \ref{img:HW5-StressEQV_1}, \ref{img:HW5-StressEQV_2} and \ref{img:HW5-StressEQV_3}, it referred to as the stress shifts from the cut boundaries to the fillet.
\begin{figure}[!h]
\centering
\subfloat[][\scriptsize Submodel at cut boundaries $65\,mm$, $65\,mm$\label{img:HW5_StressEqv_0}]{\includegraphics[width=.45\linewidth]{imgHW5/HW5-SolutionEQV_0}}\,
\subfloat[][\scriptsize Submodel at cut boundaries $70\,mm$, $67,50\,mm$\label{img:HW5_StressEqv_25}]{\includegraphics[width=.45\linewidth]{imgHW5/HW5-SolutionEQV_25}}
\caption{Von Mises equivalent stress}
\label{img:HW5-StressEQV_1}
\end{figure}
\begin{figure}[!h]
\centering
\subfloat[][\scriptsize Submodel at cut boundaries $75\,mm$, $70\,mm$\label{img:HW5_StressEqv_50}]{\includegraphics[width=.45\linewidth]{imgHW5/HW5-SolutionEQV_50}}\,
\subfloat[][\scriptsize Submodel at cut boundaries $80\,mm$, $72,50\,mm$\label{img:HW5_StressEqv_75}]{\includegraphics[width=.45\linewidth]{imgHW5/HW5-SolutionEQV_75}}
\caption{Von Mises equivalent stress}
\label{img:HW5-StressEQV_2}
\end{figure}
\begin{figure}[!h]
\centering
\subfloat[][\scriptsize Submodel at cut boundaries $85\,mm$, $75\,mm$\label{img:HW5_StressEqv_100}]{\includegraphics[width=.45\linewidth]{imgHW5/HW5-SolutionEQV_100}}\,
\subfloat[][\scriptsize Submodel at cut boundaries $90\,mm$, $77,50\,mm$\label{img:HW5_StressEqv_125}]{\includegraphics[width=.45\linewidth]{imgHW5/HW5-SolutionEQV_125}}
\caption{Von Mises equivalent stress}
\label{img:HW5-StressEQV_3}
\end{figure}
\newpage
\noindent It is observed in the graph, in the figure \ref{img:HW5-StressSensit}, which initially stress is high, then it takes on a downward path until it stabilizes. Increasing the size of the submodel is known that the more efforts are moving from the boundary condition in correspondence of the fitting and inside of the junction. 
\begin{figure}[!h]
\centering
    \resizebox{.8\linewidth}{!}{\begin{tikzpicture}
\pgfplotsset{cycle list={blue\\}, major grid style={dashed},}
\begin{axis}[
						legend cell align={left},
						width=15cm,
						height=10cm,
        				grid=major,
        				ylabel=Stress $Mpa$,
        				xlabel=Distance $mm$]
\addplot [thick, Apricot] table[smooth, mark=none, y=stress, x=rCutBun]{StressEQV.txt};
\addlegendentry{$\sigma_{eq}$ upon horizontal cut boundaries}
\end{axis}
\end{tikzpicture}}
    \caption{Submodel' s sensitivity upon the location of the cutting boundaries}
    \label{img:HW5-StressSensit}
\end{figure}
\newpage
\noindent At this point you can graph the stress distribution along the connection by defining a path.
Selecting the nodes present on the lines that define the fitting and which pass through the zone more stressed. Interactively define a path by selecting the nodes defining the effort that you want to analyze, equivalent Von Mises, getting the result shown in figure \ref{img:HW5-StressSensit}.
\begin{figure}[!h]
\centering
    \includegraphics[width=.8\linewidth]{imgHW5/HW5-SolutionEQV_151}
    \caption{Distribution of forces across the junction by defining a path}
    \label{img:HW5-StressSensit}
\end{figure}
\section{Command list}
%\begin{multicols}{2}
\tiny
%\lstinputlisting[language=APDL, style=apdl-modified]{CommandList-HW5.txt}
\begin{multicols}{2}
\begin{lstlisting}[language=APDL, style=apdl-modified]
!*********************
! PROBLEM HOMEWORK 4 *
!*********************

FINISH
/CLEAR,START,NEW
/FILNAME,Homework4
! >>> PARAMETERS MODEL <<<
*SET,leng_v_pipe,500
*SET,diameter_v_pipe,100
*SET,thick_v_pipe,2
*SET,leng_h_pipe,1000
*SET,diameter_h_pipe,200
*SET,thick_h_pipe,4
*SET,alpha,90
*SET,EPS,1E-3
*SET,n_div_area,3

! >>> PROPERTIES MATERIAL <<<
*SET,E_Young,210000
*SET,ni,0.3

! >>> LOAD <<<
*SET,pressure,1

! >>> PRE PROCESSING <<<
/PREP7
ET,1,SHELL181
KEYOPT,1,8,2
SECTYPE,1,shell
SECDATA,thick_h_pipe
MP,EX,1,E_Young
MP,PRXY,1,ni

ET,2,SHELL181
KEYOPT,2,8,2
SECTYPE,2,shell
SECDATA,thick_v_pipe
MP,EX,2,E_Young
MP,PRXY,2,ni

TYPE,1
SECNUM,1
MAT,1
SAVE

! >>> DEFINE RF CYLINDRICAL <<<
CLOCAL,100,CYLIN,0,diameter_h_pipe/2-leng_v_pipe/10,0,,-alpha
CSYS,0
CLOCAL,200,CYLIN,0,0,0,,,alpha

!***VERTICAL PIPE
CSYS,100
K,1,diameter_v_pipe/2,0,0
K,2,diameter_v_pipe/2,0,leng_v_pipe+leng_v_pipe/10
K,3,0,0,0
K,4,0,0,leng_v_pipe
L,1,2
!***HORIZZONTAL PIPE
CSYS,200
K,5,diameter_h_pipe/2,-alpha,0
K,6,diameter_h_pipe/2,-alpha,leng_h_pipe/2
K,7,0,0 
K,8,0,0,leng_h_pipe/2
L,5,6
!***GEN AREA H PIPE
AROTAT,2,,,,,,8,7,-2*alpha
!***GEN AREA V PIPE
CSYS,100
AROTAT,1,,,,,,3,4,alpha
SAVE

APTN,2,3
ADELE,4,5
LDELE,10
LDELE,15
LDELE,16
SAVE

CSYS,0
WPOFFS,-50,100
WPROTA,,250
ASBW,7
WPOFFS,50,-10
WPROTA,,-250
WPROTA,,,alpha
WPOFFS,,,64.3
ASBW,3
WPROTA,,alpha
WPOFFS,,,-10
ASBW,6
SAVE

! >>> MESHING <<<
LCCAT,16,21
LCCAT,15,21
AESIZE,ALL,n_div_area

!***HORIZZONTAL PIPE
ESYS,200
TYPE,1
SECNUM,1
MSHKEY,1
MSHAPE,0,2D
AMESH,4
AMESH,5
AMESH,2
AMESH,1
SAVE

!***VERTICAL PIPE
ESYS,100
SAVE
TYPE,2
SECNUM,2
SAVE
MSHKEY,1
MSHAPE,0,2D
AMESH,7
AMESH,3
SAVE

! >>> VERIFY THE NORMAL VERSOR <<<
EPLOT
/PSYMB,ESYS,1
 
! >>> SOLUTION <<<
SAVE
SFE,ALL,,PRES,,pressure
/PBC,ALL,,1
NSEL,S,LOC,x,-EPS,+EPS 
DSYM,SYMM,x
NSEL,S,LOC,z,-EPS,+EPS 
DSYM,SYMM,z
KSEL,S,KP,,6
NSLK,S
DSYM,SYMM,y 
ALLSEL,ALL
/SOLU
SOLVE
FINISH

! >>> POSTPROCESSING <<<
/POST1
/ESHAPE,1
PLDISP,1

SHELL,MID
RSYS,SOLU
PLNSOL,S,x
PLNSOL,S,y
PRNSOL,S,comp

!*********************
! PROBLEM HOMEWORK 5 *
!*********************

FINISH
/CLEAR, START, NEW
/TITLE, HOMEWORK 5
/FILNAME,HOMEWORK5,1
! >>> PARAMETERS MODEL <<<
*DO,k,0,150,25
*SET,leng_v_pipe,490
*SET,diameter_v_pipe,100
*SET,thick_v_pipe, 2
*SET,leng_h_pipe,1000
*SET,diameter_h_pipe,200
*SET,thick_h_pipe,4
*SET,fillet,10
*SET,alpha,90
*SET,EPS,1E-3
*SET,e_lenght,1
!***PARAMETERS CUT BOUNDARES
*SET,vCutBun,35+(k/25*5))
*SET,rCutBun,65+(k/25*5)
*SET,hCutBun,(diameter_h_pipe/2)*1.1

! >>> PROPERTIES MATERIAL <<<
*SET,E_Young, 210000
*SET,ni,0.3

! >>> LOAD <<<
*SET,pressure, 1

! >>> PRE PROCESSING <<<
/PREP7
ET,1,SOLID186
MP,EX,1,E_Young
MP,PRXY,1,ni
TYPE,1
SECNUM,1
MAT,1
SAVE
\end{lstlisting}
\end{multicols}

\begin{lstlisting}[language=APDL, style=apdl-modified]
! >>> DEFINE RF CYLINDRICAL <<<
CLOCAL,100,CYLIN,0,diameter_h_pipe/2-thick_h_pipe/2,0,,-alpha
CSYS,0
CLOCAL,200,CYLIN,0,0,0,,,alpha

!***VERTICAL PIPE
CSYS,100
WPCSYS,,100
WPOFFS,0,0,-50
CYL4,0,0,(diameter_v_pipe/2)-thick_v_pipe,0,diameter_v_pipe/2,alpha,leng_v_pipe*(1+(109/490))
!***HORIZZONTAL PIPE
CSYS,200
WPCSYS,,200
CYL4,0,0,(diameter_h_pipe/2)-thick_h_pipe,-alpha,diameter_h_pipe/2,alpha,leng_h_pipe/2

VPTN,1,2
VDELE,3,4,,1
AFILLT,33,23,fillet
AL,2,14,33
AL,1,34,16
VA,5,3,6,12,7
SAVE

!***CUT VERTICAL PIPE
WPCSYS,,100
WPOFFS,0,0,vCutBun
VSBW,5,SEPO,DELETE
!***CUT HORIZZONTAL PIPE
CSYS,100
K,100,rCutBun,-(hCutBun-diameter_h_pipe/2),-hCutBun
K,101,rCutBun,hCutBun,-hCutBun
L,100,101
K,102,0,0,-diameter_h_pipe/2
K,103,0,0,diameter_h_pipe/2
L,102,103
ADRAG,5,,,,,,6
VSBA,7,2
!***JOIN SUBMODEL
VADD,3,1,6,4
VSEL,U,VOLU,,7
VDELE,ALL,,,1
ALLSEL,ALL
!***JOINS AREAS 
AADD,26,5,31,34
AADD,41,6,32,36
SAVE
\end{lstlisting}
\begin{multicols}{2}
\begin{lstlisting}[language=APDL, style=apdl-modified]
! >>> MESHING <<<
CSYS,0
MSHAPE,1,3D
MSHKEY,0
ESIZE,e_lenght
VMESH,ALL
SAVE
/VIEW,1,,,-1
/ANG,1  
/REP,FAST 

! >>> SHELL TO SOLID SUBMODELS <<<
ASEL,S,AREA,,23,25,2
NSLA,S
NWRITE,subHW5,node,,0
ALLSEL,ALL
SAVE
RESUME,'homework4','db'
/POST1
FILE,HOMEWORK4,rst 
SET,first      
cbdof,subHW5,node,,DHW5,cbdo,,,,1

RESUME,HOMEWORK5,db
/PREP7
/INPUT,DHW5,cbdo
/INPUT,DHW5,cbdo,,:cb1

! >>> SOLUTION <<<
SAVE
NSEL,S,LOC,x,-eps,+eps
DSYM,SYMM,x
NSEL,S,LOC,z,-eps,+eps
DSYM,SYMM,z
ASEL,S,AREA,,20,24,2
ASEL,A,AREA,,1
NSLA,S
SF,ALL,PRES,pressure
ALLSEL,ALL
/SOLU
SOLVE
FINISH

/VIEW,1,,,-1
/ANG,1  
/REP,FAST 
! >>> POSTPROCESSING <<<
/POST1
PLDISP,1
PLNSOL,S,Y
PLNSOL,S,X
PRNSOL,S,COMP

PLNSOL,S,X
*GET,stressmax,PLNSOL,,max
*CFOPEN,'Stress','txt',,append
 *VWRITE,k/25,vCutBun,rCutBun,stressmax
  (F20.10,F20.10,F20.10,F20.10)
*cfclos

PLNSOL,S,eqv
*GET,stressEQV,PLNSOL,,max
*CFOPEN,'stressEQV','txt',,append
 *VWRITE,k/25,vCutBun,rCutBun,stressEQV
  (F20.10,F20.10,F20.10,F20.10)
*CFCLOS

LSEL,S,LINE,,8  
LSEL,A,LINE,,2  
LSEL,A,LINE,,28
NSLL,S,1
FLST,2,263,1
FITEM,2,1   
FITEM,2,56760   
FITEM,2,56761   
FITEM,2,56762   
FITEM,2,56763   
FITEM,2,56764   
FITEM,2,56765   
FITEM,2,56766   
FITEM,2,56767   
FITEM,2,56768   
FITEM,2,56769   
FITEM,2,56771   
FITEM,2,56771   
FITEM,2,56770   
FITEM,2,56771   
FITEM,2,56771   
FITEM,2,56771   
FITEM,2,56772   
FITEM,2,56773   
FITEM,2,56774   
FITEM,2,56775   
FITEM,2,56776   
FITEM,2,56777   
FITEM,2,56779   
FITEM,2,56779   
FITEM,2,56778   
FITEM,2,56779   
FITEM,2,56780   
FITEM,2,56781   
FITEM,2,56782   
FITEM,2,56783   
FITEM,2,56784   
FITEM,2,56785   
FITEM,2,56787   
FITEM,2,56786   
FITEM,2,56787   
FITEM,2,56786   
FITEM,2,56786   
FITEM,2,56787   
FITEM,2,56788   
FITEM,2,56789   
FITEM,2,56790   
FITEM,2,56791   
FITEM,2,56792   
FITEM,2,56793   
FITEM,2,56794   
FITEM,2,56795   
FITEM,2,56796   
FITEM,2,56797   
FITEM,2,56798   
FITEM,2,56799   
FITEM,2,56800   
FITEM,2,56801   
FITEM,2,56802   
FITEM,2,56803   
FITEM,2,56804   
FITEM,2,56805   
FITEM,2,56806   
FITEM,2,56807   
FITEM,2,56808   
FITEM,2,56809   
FITEM,2,56810   
FITEM,2,56811   
FITEM,2,56812   
FITEM,2,56813   
FITEM,2,56814   
FITEM,2,56815   
FITEM,2,56816   
FITEM,2,56817   
FITEM,2,56818   
FITEM,2,56819   
FITEM,2,56821   
FITEM,2,56821   
FITEM,2,56820   
FITEM,2,56821   
FITEM,2,56822   
FITEM,2,56823   
FITEM,2,56824   
FITEM,2,56825   
FITEM,2,56826   
FITEM,2,56827   
FITEM,2,56828   
FITEM,2,56829   
FITEM,2,56830   
FITEM,2,56831   
FITEM,2,56832   
FITEM,2,56833   
FITEM,2,56834   
FITEM,2,56835   
FITEM,2,56836   
FITEM,2,56837   
FITEM,2,56838   
FITEM,2,56839   
FITEM,2,56840   
FITEM,2,56841   
FITEM,2,56842   
FITEM,2,56843   
FITEM,2,56845   
FITEM,2,56845   
FITEM,2,56844   
FITEM,2,56845   
FITEM,2,56846   
FITEM,2,56847   
FITEM,2,56848   
FITEM,2,51894   
FITEM,2,51925   
FITEM,2,51924   
FITEM,2,51923   
FITEM,2,51922   
FITEM,2,51921   
FITEM,2,51919   
FITEM,2,51919   
FITEM,2,51920   
FITEM,2,51919   
FITEM,2,51918   
FITEM,2,51917   
FITEM,2,51916   
FITEM,2,51914   
FITEM,2,51915   
FITEM,2,51914   
FITEM,2,51914   
FITEM,2,51914   
FITEM,2,51915   
FITEM,2,51915   
FITEM,2,51913   
FITEM,2,51914   
FITEM,2,51913   
FITEM,2,51914   
FITEM,2,51914   
FITEM,2,51913   
FITEM,2,51912   
FITEM,2,51911   
FITEM,2,51910   
FITEM,2,51909   
FITEM,2,51909   
FITEM,2,51909   
FITEM,2,51907   
FITEM,2,51907   
FITEM,2,51909   
FITEM,2,51909   
FITEM,2,51907   
FITEM,2,51907   
FITEM,2,51909   
FITEM,2,51907   
FITEM,2,51909   
FITEM,2,51907   
FITEM,2,51909   
FITEM,2,51909   
FITEM,2,51909   
FITEM,2,51909   
FITEM,2,51909   
FITEM,2,51909   
FITEM,2,51907   
FITEM,2,51907   
FITEM,2,51907   
FITEM,2,51907   
FITEM,2,51908   
FITEM,2,51907   
FITEM,2,51906   
FITEM,2,51905   
FITEM,2,51904   
FITEM,2,51903   
FITEM,2,51902   
FITEM,2,51902   
FITEM,2,51902   
FITEM,2,51901   
FITEM,2,51900   
FITEM,2,51899   
FITEM,2,51898   
FITEM,2,51897   
FITEM,2,51896   
FITEM,2,51895   
FITEM,2,30418   
FITEM,2,30660   
FITEM,2,30659   
FITEM,2,30658   
FITEM,2,30657   
FITEM,2,30656   
FITEM,2,30655   
FITEM,2,30654   
FITEM,2,30653   
FITEM,2,30652   
FITEM,2,30651   
FITEM,2,30650   
FITEM,2,30649   
FITEM,2,30648   
FITEM,2,30647   
FITEM,2,30645   
FITEM,2,30645   
FITEM,2,30646   
FITEM,2,30645   
FITEM,2,30645   
FITEM,2,30645   
FITEM,2,30644   
FITEM,2,30643   
FITEM,2,30642   
FITEM,2,30641   
FITEM,2,30640   
FITEM,2,30639   
FITEM,2,30638   
FITEM,2,30636   
FITEM,2,30636   
FITEM,2,30637   
FITEM,2,30635   
FITEM,2,30635   
FITEM,2,30636   
FITEM,2,30634   
FITEM,2,30636   
FITEM,2,30633   
FITEM,2,30634   
FITEM,2,30633   
FITEM,2,30636   
FITEM,2,30635   
FITEM,2,30634   
FITEM,2,30633   
FITEM,2,30632   
FITEM,2,30631   
FITEM,2,30630   
FITEM,2,30629   
FITEM,2,30627   
FITEM,2,30627   
FITEM,2,30628   
FITEM,2,30627   
FITEM,2,30626   
FITEM,2,30625   
FITEM,2,30623   
FITEM,2,30623   
FITEM,2,30624   
FITEM,2,30622   
FITEM,2,30622   
FITEM,2,30623   
FITEM,2,30622   
FITEM,2,30621   
FITEM,2,30620   
FITEM,2,30618   
FITEM,2,30618   
FITEM,2,30620   
FITEM,2,30620   
FITEM,2,30619   
FITEM,2,30618   
FITEM,2,30616   
FITEM,2,30616   
FITEM,2,30618   
FITEM,2,30618   
FITEM,2,30618   
FITEM,2,30618   
FITEM,2,30617   
FITEM,2,30616   
FITEM,2,30615   
FITEM,2,30614   
FITEM,2,30613   
FITEM,2,30612   
FITEM,2,30611   
FITEM,2,30610   
FITEM,2,30609   
FITEM,2,30608   
FITEM,2,30607   
FITEM,2,30606   
FITEM,2,30605   
FITEM,2,30604   
FITEM,2,30603   
FITEM,2,30602   
FITEM,2,1126 
PATH,STREQV,263,30,20,  
PPATH,P51X,1
PATH,STAT     
AVPRIN,0, ,  
PDEF,VONMISES,S,EQV,AVG 
/PBC,PATH, ,1   
PLPATH,VONMISES
FINISH
/CLEAR,START
*ENDDO
\end{lstlisting}
\end{multicols}
\normalsize
%\end{multicols}